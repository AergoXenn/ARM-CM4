\documentclass[11pt]{beamer}

\usetheme{Copenhagen}

\usepackage[utf8]{inputenc}
\usepackage[spanish]{babel}
\usepackage{hyperref}

\title{Sistemas Embebidos con el ARM-Cortex-M4}
\subtitle{Conceptos Esenciales de Sistemas Embebidos}
\author[Alex S. Kelly]{Alejandro S. Kelly}
\institute[]{Pulsar Labs}


\begin{document}

    \frame{\titlepage}

    \section{Introducción}

    \begin{frame}
        \frametitle{¿Que es un sistema embebido?}

        \onslide<2>{Un Sistema Embebido es un dispositivo \emph{computarizado} dedicado realizar un conjunto de tareas \emph{tangibles} en el mundo real.\hfill \break}
            
    \end{frame}

    \begin{frame}
        \frametitle{Computación Embebida vs Computación de Propósito General}

        \only<1>{Estudiando Sistemas Embebidos, es de interés contrastar 2 paradigmas, la computación embebida (obviamente) y la general. \hfill \break}

        \only<2>{Los dispositivos de propósito general (PC's, teléfonos, etc) proveen un entorno computacional abierto y extensible por el usuario, con
                 recursos de hardware abundantes que se utilizan sin limitaciones evidentes. Esto les permite realizar infinidad de tareas.}

        \only<3>{En contraste, los sistemas embebidos proveen una funcion especifica y no modificable, el hardware es limitado, y contempla lo mínimo necesario para cumplir
                 con una función o tarea dada.
                }

    \end{frame}

\end{document}
